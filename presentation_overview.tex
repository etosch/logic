\documentclass{article}
\usepackage{fullpage}
\begin{document}
\section{Topic}
Automated, interactive theorem provers such as Coq connect applied concepts in computer science to the theory behind them. Although students often hear that logic is foundational to computer science, they rarely have the opportunity to appreciate the connections between logic and applied computer science in an introductory logic class. 

This project was an attempt to facilitate proving logical theorems in an automated theorem prover. Our goal was to write sufficient infrastructure so that students could easily write automated versions of the proofs they did in their homework. 
\section{Status}
Implementing sufficient infrastructure to help students write even simple proofs proved to be quite challenging. Although our goal was to be able to encode just the first homework assigned to CS513 in Coq, after months of trying, we were not able to do so.

We have accomplished the following:
\begin{itemize}
\item {\bf Encoded a simple logical language.} The syntax for a logical language, according to Ch.1 of Schoening, was straightforward. We defined an atomic type, denoted by \verb|A|, and indexed by the natural numbers. Formulae are defined to be atoms in the base case, disjunctions of formulae, or the negation of a formula. Syntactically incorrect formulae will not pass the type checker. 

An example of a formula is \verb|Disjunction (Atom (A 1)) (Negation (Atom (A 2)))|. This formula would appear in usual notation as $A_1 \vee \neg A_2$. Although it would be nice tfor students to be able to express these formulae in a notation that looks more like the notation they've seen in class, such substitutions as \verb|/\| for $\wedge$ are already used by Coq's Logic library, which we use in our theorem proving. This kind of aesthetic change could be made in a layer between where students enter their logical formulae and where we execute our code, to avoid notation clashes.

\item {\bf Generated a truth table and defined an evaluation function.} 
\item {\bf Encoded basic definitions for assignments, suitability, satisfiability, etc.}  
\end{itemize}

\section{Missing Pieces}
\section{Difficulties}
\section{What we learned}
\end{document}
