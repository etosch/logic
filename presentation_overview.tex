\documentclass{article}
\usepackage{fullpage}
\usepackage{todo}
\begin{document}
\section{Topic}
Automated, interactive theorem provers such as Coq connect applied concepts in computer science to the theory behind them. Although students often hear that logic is foundational to computer science, they rarely have the opportunity to appreciate the connections between logic and applied computer science in an introductory logic class. 

This project was an attempt to facilitate proving logical theorems in an automated theorem prover. Our goal was to write sufficient infrastructure so that students could easily write automated versions of the proofs they did in their homework. 
\section{Status}
Implementing sufficient infrastructure to help students write even simple proofs proved to be quite challenging. Although our goal was to be able to encode just the first homework assigned to CS513 in Coq, after months of trying, we were not able to do so.

We have accomplished the following:
\begin{itemize}
\item {\bf Encoded a simple logical language.} The syntax for a logical language, according to Ch.1 of Schoening, was straightforward. We defined an atomic type, denoted by \verb|A|, and indexed by the natural numbers. Formulae are defined to be atoms in the base case, disjunctions of formulae, or the negation of a formula. Syntactically incorrect formulae will not pass the type checker. 

An example of a formula is \verb|Disjunction (Atom (A 1)) (Negation (Atom (A 2)))|. This formula would appear in usual notation as $A_1 \vee \neg A_2$. Although it would be nice tfor students to be able to express these formulae in a notation that looks more like the notation they've seen in class, such substitutions as \verb|/\| for $\wedge$ are already used by Coq's Logic library, which we use in our theorem proving. This kind of aesthetic change could be made in a layer between where students enter their logical formulae and where we execute our code, to avoid notation clashes.

\item {\bf Generated a truth table and defined an evaluation function.} 
\item {\bf Encoded basic definitions for assignments, suitability, satisfiability, etc.}  
\end{itemize}

\section{Missing Pieces}
\section{Difficulties}
The main challenge in Coq has been to choose the right definitions in our system so that we have sufficient pieces of evidence to build an argument of correctness. If we fail to define a concept in the right way, we could find ourselves stuck in a proof. 

One example where this has happened, which we have not yet been able to resolve, has been in the definition of suitability. In our first pass, we defined using \verb|eval_formula|. \verb|eval_formula| was a function that took a formula ($\varphi$) and an assignment ($\mathcal{A}$) as arguments and returned an \verb|Option| type parameterized by booleans. If the assignment was suitable for $\phi$, \verb|eval_formula| would return the result of $\mathcal{A}(\varphi)$ as a boolean option. Suitability was defined as follows:
\\
\verb| Definition suitable (f : formula) (a : assignment) := eval_formula f a <> None.|\\
We found two issues with this approach. The first problem was that our proofs involving suitability had a lot of cruft in them. For any non-trivial proof using induction, we would need to consider the case where \verb|eval_formula| returned \verb|None|. The second problem appeared later, when we tried to prove lemmas about formulae on our way to proving homework problems. We found that many inductive proofs over \verb|eval_formula| would get ``stuck'' because the we couldn't use structural information we knew about evaluation in order to complete the proof. We tried proving auxiliary lemmas to get around this problem, hoping that they would provide sufficient rewrite rules and evidence to make progress. They did not.

Since we were getting stuck due to the ``computational'' nature of \verb|eval_formula|, we next tried to define it in an inductive, ``relational'' way. \Todo[Fix]{put this in a better way that will make neil happy}. The goal was to define suitability as something that could be built from existing knowledge of suitability. The problem we ran into was that we needed to define suitability both in terms of the assignment and the formula. We could not simply have statements such as \verb|forall phi a, suitable phi a -> suitable (Negation phi) a| or \verb|forall phi a (h : atomic * bool), suitable phi a -> auitable phi (h::a)|. We tried combining both inductive statements, as well as proving that such properties hold in one definition or the other. However, we'd once again get stuck.

Our next attempt at defining suitability returned to the ``computational'' definition, but did not rely on \verb|eval_formula|. We defined an assignment as suitable for a formula when the set of atoms in the formula are a subset of the set of atoms in the assignment. We attempt to define suitability this way much earlier, but lacked sufficient understanding of Coq to know how to use things like the Set library. 



\section{What we learned}
\end{document}
